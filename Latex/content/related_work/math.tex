\section{Math}

In case you need to include some math,
the \emph{amsmath} package\footnote{\url{https://texdoc.net/texmf-dist/doc/latex/amsmath/amsmath.pdf}} is already included in this document.

To properly display some short formula like \( e^{i \pi} = -1\),
you can use the \verb+\( \)+ inline command.
For larger formulas, the \verb+math+ environment is more appropriate.
If you need to reference the formula multiple times,
e.g. in case it is used in theorems,
you should use the \verb+equation+ environment:

\begin{equation}
	\vec\nabla\times\vec{B}= \mu_0\vec{j}+\mu_0\varepsilon_0\frac{\partial\vec{E}}{\partial t}
	\label{func}
\end{equation}

To reference it as~\ref{func} using the \verb+\ref{}+ command,
remember to use a \verb+\label{}+.
