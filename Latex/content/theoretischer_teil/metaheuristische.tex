\section{Metaheuristische Algorithmen}
Metaheuristische Algorithmen werden angewendet auf "Harte" Optimierungsprobleme.
In diese Kategorie fallen unter anderem Probleme für die kein Algorithmus 
bekannt ist, der ein globales Optimum in endlicher Zeit findet.~\cite*{metaheuristics}
Das bestimmen optimaler Parameter eines Hidden Markov Models ist demnach ein hartes Problem.

Heuristischer Algorithmus ist salopp gesagt schlaues Raten.

Vorteile Gegenüber klassischen iterativen Algorithmen
- 
- Metaheuristische Optimierungsverfahren können vielverpsrechend sein, wenn wir 
keine hochoptimierte Lösung suchen, sondern eine Lösung welche "gut genug"
ist, aber einfach zu berechnen ist. \cite{metaheuristics&evolcomp}

- Eine Metaheuristik ist nicht ein bestimmter Algorithmus sondern vielmehr 
eine Ansammlung von Ideen, Konzepten und Operatoren, welche verwendet werden können 
um einen heuristischen Algorithmus zu erstellen.~\cite*{MetaheuristicsExposed}

Eine Metaheuristik bietet ein Problemunabhängiges Framework, aus welchem man 
sich einen Heuristischen Algorithmus für ein spezifisches Problem bastelt. So muss 
man nicht ständig das Rad neu erfinden.

\section{Funktionsweise}
- heuristische Algorithmen lassen eine verschlechterung einer Lösung zu
in der Hoffnung dadurch lokalen Minima zu entkommen. \cite*{metaheuristics}

% Motivation
Metaheuristische Algorithmen in letzter zeit am stabbo

% Was ist ein Metaheuristischer Algorithmus
Ein Metaheuristischer Algorithmus ist inspiriert durch die dynamiken in einem Biologischen, Physikalischen
oder auch Sozialen System.

% Beisipele
Bekannte Beipsiele sind Particle Swarm Optimization, Genetic Algorithm 
und 

% Arten von Metaheuristischen Algorithmen
Key unterschied ob es sich um einen Single-solution oder Population based handelt



% Funktionsweise
Key komponente ist Exploration und Exploitation
Manchmal auch intensivication und diversification gennant [missing quote]

Exploitation beschreibt das verbessern einer gegebenen Lösung
Exploration beschreibt das finden neuer Lösungen

Es ist wichtig eine gute Balance zu finden, denn wenn die 
Nur Exploitation ohne Exploration ist equivalent zu Lokaler Suche

Nur Exploration ohne Exploitation ist equivalent zu random Search

\subsection{Kritik an Metaheuristischen Algorithmen}
In den letzten Jahren wurde das Feld der Optimierung regelrecht überschwemmt mit 
"neuen" Metapher basierten Algorithmen. Ob Mikrofledermaus, Jazz-Musiker, Schwarze Löcher
oder auch intelligente Wassertropfen. Jedes erdenkliche Konzept kann in einen 
Optimierungsalgorithmus verwandelt werden.
Oft stellt sich jedoch heraus, dass das einzige was diese "neuen" Algorithmen zum Feld beitragen
eine Umbenennung eines bereits etablierten Algorithmus ist.~\cite*{NoNovelty}
So ist zum Beispiel Harmony search, ein Suchalgorithmus der auf dem Prinzip von 
Jazz-Musikern funktioniert nichts weiteres als eine Umbenennung eines 
speziellen Falles des Genetischen Algorithmus.~\cite*{HarmonySearch}
Trotz mangelnder Innovation verzeichnet eine Suche nach "harmony search"
auf Google Scholar laut Weyland im Jahre 2010 586 Einträge. Im Jahre 2023 ist diese Zahl 
auf stolze 57.500 Einträge gestiegen, wovon 7.840 Einträge nach 2022 erschienen.
Die Flut solcher vermeintlich "neuen" Algorithmen ist sehr nachteilig für das 
Feld der Optimierung, denn die elaborierten Metaphern für bereits existierende Konzepte 
führen zu Verwirrung und tatsächlich innovative Ansätze werden übersehen.~\cite*{MetaheuristicsExposed}