\section{Metaheuristiken}
Bei einer Metaheuristik handelt es sich nicht um einen bestimmten Algorithmus sondern vielmehr eine Ansammlung von Ideen, Konzepten und Operatoren, welche verwendet werden können um einen spezifischen heuristischen Algorithmus zu erstellen \cite*{MetaheuristicsExposed}. Es gibt also zum Beispiel nicht \textit{den} genetischen Algorithmus sondern es existiert ein Genetischer Algorithmus "Bauplan" an welchem man sich orientieren kann.

Um eine Metaheuristik zu erstellen gilt es zwei konfliktierende Kriterien zu balancieren. Zum einen die \textbf{Diversifizierung} (exploration): das entdecken neuer Lösungen und zum anderen die \textbf{Intensivierung} (exploitation): das verbessern einer bekannten Lösung \cite*{MetaheuristicsEGT}. Reine Diversifizierung ist das selbe wie eine zufällige Suche und reine Intensivierung ist equivalent zu einer lokalen Suche.

Metaheuristiken sind oft inspiriert durch natürliche Prozesse. Bekannte Metaheuristiken aus dem Gebiet der Biologie sind \textbf{Genetische Algorithmen (GA)}, welche die Evolution einer Population durch natürliche Selektion nachahmen und \textbf{Ant Colony Optimization (ACO)} Algorithmen, welche der Pheromon-basierten Kommunikation von Ameisen innerhalb einer Kolonie nachempfunden sind. Zu den bekanntesten Beispielen aus der Physik zählt das \textbf{Simulated Annealing (SA)} Verfahren, welches angelehnt ist an den Abkühlungsprozess eines Metalls nach dem erhitzen \cite*{metaheuristics}.

\subsection*{Hybride Heuristiken}
Eine kombination mehrerer verschiedener Algorithmen zu einem neuen nennt man Hybridisierung. Das kombinieren von metaheuristischen Algorithmen ist ein aktives Forschungsfeld und es existiert mittlerweile eine beachtliche Anzahl hybrider Metaheuristiken. Das kombinieren von Konzepten erlaubt es Wissenschaftlern außerhalb der Grenzen einer bestimmten Metaheuristik zu denken und ihrer Kreativität freien Lauf zu lassen \cite*{MetaheuristicsSurvey}.

Es gibt verschiedene möglichkeiten der Hybridisierung. Wenn man verschiedene Heuristiken sequentiell hintereinander ausführt spricht man von einer \textbf{Relay Hybridization (RH)}. Eine Hybridisierung in der verschiedene Heuristiken kooperativ zusammenarbeiten nennt man eine \textbf{Teamwork Hybridization (TH)} \cite*{MetaheuristicsEGT}. 

Die Hauptmotivation hinter der hybridisierung ist es komplementäre Eigenschaften verschiedener Algorithmen auszunutzen \cite*{MetaheuristicsSurvey}. Populationsbasierte Metaheuristiken, wie zum Beispiel genetische Algorithmen sind gut im diversifizieren aber schlecht im intensivieren. Eine lokale Suche wiederum ist gut im intensivieren einer Lösung, bietet aber keine Diversifizierung. Es bietet sich also an Eine populationsbasierte Metaheuristik mit einer lokalen Suche zu kombinieren \cite*{MetaheuristicsEGT}.

\subsection*{Parameter Tuning}
Ein großer Nachteil vom Metaheuristiken ist, dass sie neue Parameter einführen, welche selbst optimiert werden müssen. Beispiele für solche Parameter sind die Mutationsrate eines Genetischen Algorithmus, die initiale Temperatur beim simulated annealing oder auch die Pheromonpersistenz eines Ant Colony Algorithmus. Eine Optimale Belegung dieser Parameter ist Problemabhängig. Daher gibt es keine Metaheuristik für welche universal optimale Parameter existieren. Man unterscheidet zwischen dem \textbf{Off-Line} und dem \textbf{On-Line} Parameter Tuning. In einem Off-Line Ansatz werden Werte für die Parameter vor der Ausführung des Algorithmus fixiert. Beim On-Line Parameter Tuning werden die Parameter während der Ausführung dynamisch angepasst \cite*{MetaheuristicsEGT}.

\subsection*{Kritik an Metaheuristiken}
Nach dem Ansturm an Metaheuristiken welchen wir in den letzten Jahren beobachten konnten hagelt es nun auch vereinzelt Kritik zur Verwendung dieser. Besonders hybride Metaheuristiken machen oft den Anschein, dass sich die Ersteller unzureichend bis gar nicht mit der Materie außeinandergesetzt haben.
"Unfortunately, the used research methodology is often
characterized by a rather ad hoc approach that consists in mixing different algorithmic components without any really serious
attempts to identify the contribution of different components to the
algorithms' performance" \cite*{MetaheuristicsSurvey}

So gibt es zum Beispiel viele Algorithmen welche eine hybridisierung zweier stochastischer populationsbasierter Metaheuristiken sind. Da die beiden Metaheuristiken sich nicht komplementieren ist das Rational hinter solch einer Hybridisierung schwer nachvollziehbar.

Vorallem die zunehmend absurder werdenden Metaphern hinter "neuen" Metaheuristiken ernten immer mehr Kritik. Ob Kakerlaken Befall \cite*{MetaheuristicsExampleRoachInfestation}, Jazz-Musiker \cite*{MetaheuristicsExampleHarmonySearch}, Schwarze Löcher \cite*{MetaheuristicsExampleBlackHole} oder auch intelligente Wassertropfen \cite*{MetaheuristicsExampleIntelligentWaterDrops}. Jedes erdenkliche Konzept kann in einen Optimierungsalgorithmus verwandelt werden. Oft stellt sich jedoch heraus dass das einzige was diese ,,neuen" Algorithmen zum Feld beitragen eine Umbenennung eines bereits etablierten Algorithmus ist \cite*{NoNovelty}. So ist zum Beispiel Harmony search, ein Suchalgorithmus der auf dem Prinzip von Jazz-Musikern funktioniert nichts weiteres als eine Umbenennung eines speziellen Falles des Genetischen Algorithmus \cite*{HarmonySearch}. Trotz mangelnder Innovation verzeichnet eine Suche nach ,,harmony search" auf Google Scholar laut Weyland im Jahre 2010 586 Einträge. Im Jahre 2023 ist diese Zahl auf stolze 57.500 Einträge gestiegen, wovon 7.840 Einträge nach 2022 erschienen. Die Flut solcher vermeintlich "neuen" Algorithmen ist sehr nachteilig für das Feld der Optimierung, denn die elaborierten Metaphern für bereits existierende Konzepte führen zu Verwirrung und tatsächlich innovative Ansätze werden übersehen \cite*{MetaheuristicsExposed}.