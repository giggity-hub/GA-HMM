\section*{Vergleich genetischer Operatoren}
Zunächst werden wir wir die verschiedenen Vorgeschlagenen genetischen Operatoren miteinander vergleichen bezüglich der Kapazität eine gegebene Lösung zu verbessern. Dazu werden Crossover und Mutationsoperatoren auf Chromosome mit verschiedenen Optimierungsgrad angewendet. Wir können beobachten, dass sowohl Mutation als auch Crossover nur in den seltensten Fällen tatsächlich zu einer Verbesserung führen. Das zeigt, dass wir bei einem genetischen Algorithmus den größten Teil der Rechenleistung darauf verwenden Lösungen zu verschlechtern.

\section*{Performance Vergleich}
In diesem Abschnitt vergleichen wir die Performance eines genetischen Algorithmus mit dem Baum-Welch Algorithmus und betrachten welche Schritte des GA-HMM am meißten Rechenzeit beanspruchen. Typischerweise ist bei einem genetischen Algorithmus die Fitnessfunktion am Rechenintensivsten \cite*{MetaheuristicsEGT}. Im Fall des GA-HMM ist die Fitness-Funktion jedoch sehr günstig. Es muss hier angemerkt werden, dass es sich nicht um fairen Vergleich handelt, da die Fitness-Funktion in optimierten C Code implementiert ist und der genetische Teil des GA-HMM nur Teilweise optimiert ist. Ich bezweifle jedoch, dass man die Performance des genetischen Algorithmus stark verbessern kann ohne die Lesbarkeit oder Flexibilität des Codes zu komprimieren. Ein genetischer Algorithmus muss enorm viele Parameter festhalten und wurde daher als Objekt implementiert, welche zumindestens in Python sehr ineffizient sind. Der Baum-Welch Algorithmus hingegen kann als eine einizige Funktion implementiert werden, welche leicht optimiert werden kann. 

\section*{Implementation verschiedener Ansätze}
Ja sind alle kacke wer hätte gedacht?

Richard S. Barr schreibt, dass die Aussagekräftigste Visualiserung eine Darstellung der Qualität gegen Zeit ist.
Perhaps the most popular and illuminating exhibits of heuristic performance is a graph 
of solution quality as a function of time.~\cite*{ComparisonGuidelines}
Daher macht es mich stutzig warum gerade solch eine Visualisierung aus nicht in den Evaluationen der genetischen Algorithmen zu finden ist.


