\chapter*{Abstract}
Hidden Markov Modelle sein ein weit verbreitetes Werkzeug für Signalmodellierung. Die Parameter eines HMM werden typischerweise mit dem Baum-Welch Algorithmus trainiert, welcher jedoch nur lokale Optima findet. Der Baum-Welch Algorithmus wurde zu Begin der 1970er Jahre entwickelt und ist somit über 50 Jahre alt. In dieser Zeit wurden viele alternative Trainingsverfahren entwickelt mit dem Ziel eine bessere globale Suchkapazität zu bieten. Einige dieser Algorithmen sind sogenannte Metaheuristiken welche oft den Baum-Welch Algorithmus hybridisieren. Ziel dieser Arbeit ist es zu evaluieren in wie weit solche (hybriden) Metaheuristiken geeignet sind für die Parameter-Estimation eines HMM, wobei wir uns in der Auswertung auf genetische Algorithmen beschränken, welche zu den meißterforschten Metaheuristiken zählen. Zu diesem Zweck wurde ein Framework für das trainieren von HMMs mit genetischen Algorithmen entwickelt, mit welchem drei verschiedene Ansätze untersucht werden. Die Auswertung zeigt, dass genetische Algorithmen nicht geeignet sind für das Trainieren von HMMs und das der Baum-Welch Algorithmus nicht ohne Grund weiterhin der Standard für die Parameter-Estimation ist.