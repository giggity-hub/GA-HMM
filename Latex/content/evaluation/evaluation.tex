
Perhaps the most popular and illuminating exhibits of heuristic performance is a graph 
of solution quality as a function of time.~\cite*{ComparisonGuidelines}
- Wird Fragen auf warum gerade diese Darstellung in vielen "neuen" Algorithmen 
abstinent ist

When testing an algorithm that finds an optimal solution to 
a given problem, the important issues are speed and rate of convergence to the optimal 
solution.~\cite*{ComparisonGuidelines}
- Ein Genetischer Algorithmus ist weder schnell, noch schlägt er den Baum Welch Algorithmus in Konvergenz



Welche Kriterien kann man evaluieren:
- Geschwindigkeit
- Konvergenz rate


Probleme der Evaluation von "neuen" HMM trainings Algorithmen:
- Für viele Probleme ist eine Optimale Lösung nicht bekannt


Laut Barr sollten folgende Werte für Performance erhoben werden:
- Zeit bis zur besten Lösung,
- Totale Rechenzeit,
- Zeit pro Phase (in diesem Fall, crossover, selection, mutation, bw)


\subsection*{Robustheit}
Ein Algorithmus, welcher ein gutes Ergebnis nur für ein spezifisches Problem demonstrieren kann
ist nicht von großem Interesse.
Leider beschränken sich die meißten Algorithmen auf ein spezifisches realweltiches Problem oder 
sogar nur ein Toy-Example
- Bei Qualität over Time standard deviation mit plotten

Die Performance sollte evaluiert werden bevor die Parameter getuned werden.
- Auch hier testen viele Larrys ihre Algorithmen nur mit einer Konfiguration von hyper parametern

\subsection*{Ein fairer Vergleich}


% 4 States
% - Only even symbols
% - Only odd symbols
% - Only Powers of two
% - 